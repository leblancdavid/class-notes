

\title{Jeshuran Factor}
\author{
}
\date{\today}

\documentclass[11pt]{article}

\usepackage{geometry}
\geometry{
	a4paper, 
	total={8.5in,11in},
	left=0.5in,
	right=0.5in,
	top=0.5in,
	bottom=0.5in,
}

\begin{document}
\maketitle

\section{Class 1}
The gospel center around the kingdom offered to Israel. During the church age, the judgment of Israel is postponed until the rapture.

The beatitudes (Matthew 5:11-16):

\begin{enumerate}
	\item Verse 11 has a transition towards the disciples, they are addressed directly.
	\begin{enumerate}
		\item Challenge to the disciples to become part of the pivot to be the preserving factor.
		\item Believers who advance and reach maturity are part of the pivot through volition.
		\item Result is blessing and preservation of the nation, and opportunity of the gospel.
	\end{enumerate}

	\item Verse 12: The disciples reward in heaven
	\begin{enumerate}
		\item Blessings and opportunities who will become available to the disciples
		\item NOT blessings on earth, but in heaven, not material rewards
		\item Will come into reality when the church comes into history, the church age
		\item Rewards of the church age will surpass all the rewards in history
	\end{enumerate}

	\item Verse 16: See the good works and glorify the Father
	\begin{enumerate}
		\item The disciples did not understand the relationship with the Father, since Jesus had not yet been glorified
		\item Only way to be called a child of God is to be identified with Christ
		\item Jesus Christ introduced the Father, he was mentioned in the old testament but was never introduced
		\begin{enumerate}
			\item Introduction of a brand new relationship in the new testament
			\item No other people in history have the relationship with God we have today
			\item Prophetic fulfillment of a new relationship for the disciples
			\item Jesus claimed to be the Son of God, which made him in equal standing with God the Father
			\item Based on the hypostatic union, the basis of our relationship with God
			\item Did not come into effect until the day of Pentacost when the Holy Spirit was poured on the believers 
		\end{enumerate}
	\end{enumerate}

	\item The pivot is comprised of mature believes, sometimes many, sometimes few:
	\begin{enumerate}
		\item Moses was an example, Aaron was a weak example of a believer
		\item Moses was allowed to see the land but not enter
		\item We have the Jeshuran Fraternity factor here in this country, desire to continue on through spiritual maturity
		\item Today's ministry is invisible impact
		\begin{enumerate}
			\item Someone who understand the lifestyle and doctrine
			\item Living the unique spiritual life of the believer
			\item Standing for sound doctrine
			\item Invisible to us, but not to God or the Spiritual Realm
		\end{enumerate}
		\item Unique fraternity of mature believers
		\begin{enumerate}
			\item Have come to believe that bible doctrine is the most important
			\item Becomes the gem of history, and the preservers of the client nations to God
			\begin{enumerate}
				\item Nations founded on biblical principles
				\item Support missionary efforts
				\item Havens for the Jewish people
				\item Support sound biblical teaching
			\end{enumerate}
			\item The name of this group is Jeshuran
			\begin{enumerate}
				\item Only found about 25 times in the old testament
				\item Moses coined the term
				\item Means to go in a straight line, blameless, to be righteous in the spiritual sense, perfect integrity
				\item Spiritual life in compatibility with the Righteousness of God
				\item The righteous ones, the upright ones
			\end{enumerate}
		\end{enumerate}
	
		\item Enter into the fraternity through spiritual maturity
		\begin{enumerate}
			\item Doctrine becomes part of the life
			\item Practice, and consistency, muscle memory
			\item Taking in the Word of God and applying it consistently
			\item Follow and trust the pattern, lifestyle of confidence, responding to the love of God
			\item Passing the testing of your life, purpose, plan
			\item Consistent rebound when there is interruption to the progress to maturity, spiritual momentum 
		\end{enumerate}
	
		\item Harmonious rapport with God
		\begin{enumerate}
			\item Regardless of the circumstances and situation, trust in God
			\item In God's power, not your own
			\item Not guaranteed, it is a result
		\end{enumerate}
		
	\end{enumerate}
\end{enumerate}



\section{Class 2}
\begin{enumerate}
	\item There is a small group of Jeshuran believers in any generation, the pivot
	\begin{enumerate}
		\item On this basis, God preserves the generation that fails for the next generation
		\item Living in every generation throughout history
		\item Ends on the fifth cycle of discipline
		\item Fullness of blessing from God, a gift of God
		\item God provides everything necessary to continue spiritual momentum
		
	\end{enumerate}
	\item Jeshuran is the highest rank of promotion in all of human history:
	\begin{enumerate}
		\item Key to maintaining this status:
		\begin{enumerate}
			\item Don't be discouraged by your own sins and failures
			\item Golf swing and muscle memory analogy: consistency, keep moving on when failure occurs
			\item Interruptions will occur but rebound to return
			\item Grace orientation, faith rest
			\item Taking in the Word of God
			\item Rebound and keep moving!
		\end{enumerate}
		\item Protected by the power of God
		\begin{enumerate}
			\item Continuous testing of status
			\item God's purpose for your life
			\item Pass the testing, maintaining of the status results in glorifying God
		\end{enumerate}
	\end{enumerate}
	\item The four references to the Jeshuran factor:
	\item Deut 32:15
	\begin{enumerate}
		\item Prophecy given by Moses concerning the Jeshuran fraternity
		\item Specifically referring to the pivot of believers in Israel
		\item \emph{Growing fate} refers to the prosperity which becomes an distraction
		\begin{enumerate}
			\item Soon, bad behavior becomes automatic, forget grace
			\item Muscle memory in the wrong direction, back to the old sin nature
			\item Mature Jeshuran believers are NOT exempt for this pattern
			\item Circumstances can cause distractions
		\end{enumerate}
		\item \emph{Kicking} representing disobedience and rebellion against God
		\item Must be careful to not get caught up in the idolatry of the age or the pursuit of happiness
		\begin{enumerate}
			\item Blessings and happiness come from God, not our own works
			\item Frantic search for happiness is idolatry
			\item Whatever decision you make, if you are in the will of God, He will provide anything you need
		\end{enumerate}	
	\end{enumerate}
	\item Deut 33:2-5
	\begin{enumerate}
		\item Jeshuran is identified with Jesus Christ, the King of Jeshuran
		\begin{enumerate}
			\item Emphasis on His humanity
		\end{enumerate}
		\item Jesus is called the God of Jeshuran (verse 26)
		\begin{enumerate}
			\item Emphasis on Christ's Deity
			\item Emphasis on the preserving remnant of Jeshuran believers
			\item Provides protection, deliverance, blessings to continue spiritual momentum
			\item He is the only One, there's none like Him
		\end{enumerate}
		\item King and God: hypo-static union
		\begin{enumerate}
			\item Undiminished Deity and perfect humanity
		\end{enumerate}
		\item Opportunity to live in the fullness of Christ
		\item God controls history
		\begin{enumerate}
			\item What the righteousness of God approves, the blessing from God is poured down
			\item Preservation of the nation, through the pivot
			\item Don't be discouraged by the state of the nation, continue on with the spiritual momentum
			\item God does not want us to miss out on spiritual blessings He has planned or provided to us
		\end{enumerate}
	\end{enumerate}
	\item Isaiah 44:1-2
	\begin{enumerate}
		\item Used to encourage and challenge the nation of Israel
		\item Jacob: a general reference to the people
		\item Israel: Used to denote positive believers, growing believers
		\item Jeshuran: The core of mature believers, the source and target of continuation of blessing from God
		\begin{enumerate}
			\item Election to privilege of the believers
			\item Centered around the Jeshuran fraternity:
			\item Five advantages:
			\begin{enumerate}
				\item First priest or client nation
				\item Become the custodians of the covenant and Word of God
				\item A true Theocratic form of government (which they rejected twice: Once with Samuel: \emph{... give us a king like the other nations!}, and Jesus when he offered them the kingdom
				\item Become the channel for the Messiah King, the Lord Jesus Christ
				\item Destined to be the capital nation of the world, during the millennial reign of Christ, in Jerusalem (Ez 37: 27-38, 43: 4-7)
			\end{enumerate}
		\end{enumerate}
	\end{enumerate}
\end{enumerate}



\section{Class 3}
Isaiah 44 (Continued)
\begin{enumerate}
	\item Used the Jeshuran fractor to encourage God's people, and the people responded to the teaching of the Word of God, which resulted in a reprieve from judgment from God because of their idolatry and rejection of God (721 B.C.).
	\item Jeshuran is able to fulfill the election to privilege of a client nation. The Jews were the first client nation.
	\item Destined to the be capital nation of the world (the temple)
	\begin{enumerate}
		\item Ez 37:27-38
		\item Ez 43:4-7
	\end{enumerate} 
	\item Caleb and Joshua joined into the Jeshuran fraternity, they had reached maturity
	\begin{enumerate}
		\item Num 14:5-9: Refused to claim the land that God had given and promised to them
		\item Num 14:24: Jeshuran poster boy: Caleb
		\item Deut 1:4, Caleb is allowed to enter the land because he was faithful to God
	
	\end{enumerate}
	\item \emph{fear not}, fear is a distraction, interferes with spiritual growth, removed by living the spiritual life and reaching maturity
	\begin{enumerate}
		\item There is no fear in death or life in the Jeshuran factor
		\item Phil 1:21 (Paul's Jeshuran status)
		\item Converts the outward circumstances of adversity (inevitable) into stress (inward, manageable)
		\item Seven things about fear:
		\begin{enumerate}
			\item More you surrender to fear, the more things you fear
			\item Extent to surrender to fear, the greater the capacity for fear (\emph{running when no one is chasing you})
			\item The greater your capacity for fear, the more the power fear has over you
			\item The more you increase the power of fear, the more emotional sin and irrationality. 
			\item The believe who lives by fear is intimidated by life, by circumstances, by people, easily manipulated, no contentment, no poise.
			\item Fear of death does not prevent death, but will prevent one from entering the Jeshuran status, and dying in grace
			\item The status of Jeshuran is characterized by the virtue love by the believer (1John 4:8), love drives out fear 
		\end{enumerate}
	\end{enumerate}
	\item No believer attains, maintains, or advances in the Jeshuran status without the \emph{rebound} technique (1John 1:9)
	
	\item Isaiah 44:3-4: Reminder by Isaiah of Moses' sin
	\begin{enumerate}
		\item Moses' emotional sin of rejection of God caused him not to be allowed to enter the land
		\item In life God has a divine plan for everyone's life
		\item Exodus 17: Jeshuran's vulnerability, compromising the integrity of God, a warning to mature believers
		\begin{enumerate}
			\item Moses was guilty of compromising on the integrity of God
			\item Ex. 17:5, Num 20:7-9
		\end{enumerate}
		\item Deut 1:22, 26, 27, 33; Recounting the whole episode
	\end{enumerate}
\end{enumerate}



\section{Class 4}
\begin{enumerate}
	\item Deut. 1: 26-33: Apostasy, lack of trust, rebellion against the LORD
	\item Deut. 1: 37: The blame game...
	\item Moses compromising the integrity of God, striking the rock with his own rod instead of the rod that God had commanded him to. Because Moses had not \emph{believed} God, Moses is punished by not being allowed to enter into the land promised to Israel
	\begin{enumerate}
		\item He did not exercise faith rest technique, lack of patience and discipline
		\item Our mistakes in life do not affect the integrity of the plans/will of God: The result is loss of rewards and blessings from God
		\item Did not lean on or believe on the integrity of God had told him to do
		\item Did not treat God has Holy, or set apart resulting in an act of rebellion against God
	\end{enumerate}
\end{enumerate}


\section{Class 5: Jeshuran in the Church}
\begin{enumerate}
	\item The switch from the Jeshuran in the Old Testament, into the Church age in the new Testament
	\item We are \emph{in Christ}, therefore we are royalty, the Jeshuran concept in the Church age revolves around the same concepts that we have talked before, spiritual maturity and growth
	\item New Testament nomenclature
	\begin{enumerate}
		\item Eph 4:13, \emph{Telios, to a mature man}
		\begin{enumerate}
			\item Expectation of advancement in maturity
			\item \emph{fullness}: Key in the Jeshuran factor
			\item We are no longer to be like children but to mature in the faith
			\item Maturity comes as a result of volition, it is up to the believer to choose to advance into maturity
		\end{enumerate}
		\item James 1: 2-4
		\begin{enumerate}
			\item \emph{complete}, lacking in nothing, perfect and complete, understanding the council and plan of God, connecting the dots from the Word of God
			\item Understanding that the Old Testament points to the work of Christ on the cross
			\item Understanding the New Testament pointing back to the work for Christ
			\item When we become perfect and complete, we have reached \emph{Super Grace}
		\end{enumerate}
		\item Romans 4: 19, \emph{dumnimus} to grow stronger in faith
		\begin{enumerate}
			\item The challenge of the Jeshuran in the spiritual life, despite any circumstance that we find ourselves in, to grow strong in the faith, needs to be the focus
			\item Giving glory to God, and being fully assured of God's promises to provide everything we need to live the spiritual life
		\end{enumerate}
		\item Romans 15: 1
		\begin{enumerate}
			\item Living a life centered around oneself is not a trait of the Jeshuran believer
			\item Consider other people, compassion for others
		\end{enumerate}
		\item 2Tim 1:7
		\begin{enumerate}
			\item \emph{spirit} here is used in a meaning of \emph{lifestyle}
			\item Emotional intimidation, lifestyle of fear, that is the lifestyle of the unbeliever, not given to us by God
			\item Given a lifestyle of power (\emph{dumnimus}), the Word of God
			\item Holy Spirit is not grieved in the life of a Jeshuran believer with momentum, with consistency
			\item Jeshuran realm of victory, in any circumstance
		\end{enumerate}
		
	\end{enumerate}
\end{enumerate}

\end{document}